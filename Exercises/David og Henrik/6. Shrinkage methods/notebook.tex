
% Default to the notebook output style

    


% Inherit from the specified cell style.




    
\documentclass[11pt]{article}

    
    
    \usepackage[T1]{fontenc}
    % Nicer default font (+ math font) than Computer Modern for most use cases
    \usepackage{mathpazo}

    % Basic figure setup, for now with no caption control since it's done
    % automatically by Pandoc (which extracts ![](path) syntax from Markdown).
    \usepackage{graphicx}
    % We will generate all images so they have a width \maxwidth. This means
    % that they will get their normal width if they fit onto the page, but
    % are scaled down if they would overflow the margins.
    \makeatletter
    \def\maxwidth{\ifdim\Gin@nat@width>\linewidth\linewidth
    \else\Gin@nat@width\fi}
    \makeatother
    \let\Oldincludegraphics\includegraphics
    % Set max figure width to be 80% of text width, for now hardcoded.
    \renewcommand{\includegraphics}[1]{\Oldincludegraphics[width=.8\maxwidth]{#1}}
    % Ensure that by default, figures have no caption (until we provide a
    % proper Figure object with a Caption API and a way to capture that
    % in the conversion process - todo).
    \usepackage{caption}
    \DeclareCaptionLabelFormat{nolabel}{}
    \captionsetup{labelformat=nolabel}

    \usepackage{adjustbox} % Used to constrain images to a maximum size 
    \usepackage{xcolor} % Allow colors to be defined
    \usepackage{enumerate} % Needed for markdown enumerations to work
    \usepackage{geometry} % Used to adjust the document margins
    \usepackage{amsmath} % Equations
    \usepackage{amssymb} % Equations
    \usepackage{textcomp} % defines textquotesingle
    % Hack from http://tex.stackexchange.com/a/47451/13684:
    \AtBeginDocument{%
        \def\PYZsq{\textquotesingle}% Upright quotes in Pygmentized code
    }
    \usepackage{upquote} % Upright quotes for verbatim code
    \usepackage{eurosym} % defines \euro
    \usepackage[mathletters]{ucs} % Extended unicode (utf-8) support
    \usepackage[utf8x]{inputenc} % Allow utf-8 characters in the tex document
    \usepackage{fancyvrb} % verbatim replacement that allows latex
    \usepackage{grffile} % extends the file name processing of package graphics 
                         % to support a larger range 
    % The hyperref package gives us a pdf with properly built
    % internal navigation ('pdf bookmarks' for the table of contents,
    % internal cross-reference links, web links for URLs, etc.)
    \usepackage{hyperref}
    \usepackage{longtable} % longtable support required by pandoc >1.10
    \usepackage{booktabs}  % table support for pandoc > 1.12.2
    \usepackage[inline]{enumitem} % IRkernel/repr support (it uses the enumerate* environment)
    \usepackage[normalem]{ulem} % ulem is needed to support strikethroughs (\sout)
                                % normalem makes italics be italics, not underlines
    

    
    
    % Colors for the hyperref package
    \definecolor{urlcolor}{rgb}{0,.145,.698}
    \definecolor{linkcolor}{rgb}{.71,0.21,0.01}
    \definecolor{citecolor}{rgb}{.12,.54,.11}

    % ANSI colors
    \definecolor{ansi-black}{HTML}{3E424D}
    \definecolor{ansi-black-intense}{HTML}{282C36}
    \definecolor{ansi-red}{HTML}{E75C58}
    \definecolor{ansi-red-intense}{HTML}{B22B31}
    \definecolor{ansi-green}{HTML}{00A250}
    \definecolor{ansi-green-intense}{HTML}{007427}
    \definecolor{ansi-yellow}{HTML}{DDB62B}
    \definecolor{ansi-yellow-intense}{HTML}{B27D12}
    \definecolor{ansi-blue}{HTML}{208FFB}
    \definecolor{ansi-blue-intense}{HTML}{0065CA}
    \definecolor{ansi-magenta}{HTML}{D160C4}
    \definecolor{ansi-magenta-intense}{HTML}{A03196}
    \definecolor{ansi-cyan}{HTML}{60C6C8}
    \definecolor{ansi-cyan-intense}{HTML}{258F8F}
    \definecolor{ansi-white}{HTML}{C5C1B4}
    \definecolor{ansi-white-intense}{HTML}{A1A6B2}

    % commands and environments needed by pandoc snippets
    % extracted from the output of `pandoc -s`
    \providecommand{\tightlist}{%
      \setlength{\itemsep}{0pt}\setlength{\parskip}{0pt}}
    \DefineVerbatimEnvironment{Highlighting}{Verbatim}{commandchars=\\\{\}}
    % Add ',fontsize=\small' for more characters per line
    \newenvironment{Shaded}{}{}
    \newcommand{\KeywordTok}[1]{\textcolor[rgb]{0.00,0.44,0.13}{\textbf{{#1}}}}
    \newcommand{\DataTypeTok}[1]{\textcolor[rgb]{0.56,0.13,0.00}{{#1}}}
    \newcommand{\DecValTok}[1]{\textcolor[rgb]{0.25,0.63,0.44}{{#1}}}
    \newcommand{\BaseNTok}[1]{\textcolor[rgb]{0.25,0.63,0.44}{{#1}}}
    \newcommand{\FloatTok}[1]{\textcolor[rgb]{0.25,0.63,0.44}{{#1}}}
    \newcommand{\CharTok}[1]{\textcolor[rgb]{0.25,0.44,0.63}{{#1}}}
    \newcommand{\StringTok}[1]{\textcolor[rgb]{0.25,0.44,0.63}{{#1}}}
    \newcommand{\CommentTok}[1]{\textcolor[rgb]{0.38,0.63,0.69}{\textit{{#1}}}}
    \newcommand{\OtherTok}[1]{\textcolor[rgb]{0.00,0.44,0.13}{{#1}}}
    \newcommand{\AlertTok}[1]{\textcolor[rgb]{1.00,0.00,0.00}{\textbf{{#1}}}}
    \newcommand{\FunctionTok}[1]{\textcolor[rgb]{0.02,0.16,0.49}{{#1}}}
    \newcommand{\RegionMarkerTok}[1]{{#1}}
    \newcommand{\ErrorTok}[1]{\textcolor[rgb]{1.00,0.00,0.00}{\textbf{{#1}}}}
    \newcommand{\NormalTok}[1]{{#1}}
    
    % Additional commands for more recent versions of Pandoc
    \newcommand{\ConstantTok}[1]{\textcolor[rgb]{0.53,0.00,0.00}{{#1}}}
    \newcommand{\SpecialCharTok}[1]{\textcolor[rgb]{0.25,0.44,0.63}{{#1}}}
    \newcommand{\VerbatimStringTok}[1]{\textcolor[rgb]{0.25,0.44,0.63}{{#1}}}
    \newcommand{\SpecialStringTok}[1]{\textcolor[rgb]{0.73,0.40,0.53}{{#1}}}
    \newcommand{\ImportTok}[1]{{#1}}
    \newcommand{\DocumentationTok}[1]{\textcolor[rgb]{0.73,0.13,0.13}{\textit{{#1}}}}
    \newcommand{\AnnotationTok}[1]{\textcolor[rgb]{0.38,0.63,0.69}{\textbf{\textit{{#1}}}}}
    \newcommand{\CommentVarTok}[1]{\textcolor[rgb]{0.38,0.63,0.69}{\textbf{\textit{{#1}}}}}
    \newcommand{\VariableTok}[1]{\textcolor[rgb]{0.10,0.09,0.49}{{#1}}}
    \newcommand{\ControlFlowTok}[1]{\textcolor[rgb]{0.00,0.44,0.13}{\textbf{{#1}}}}
    \newcommand{\OperatorTok}[1]{\textcolor[rgb]{0.40,0.40,0.40}{{#1}}}
    \newcommand{\BuiltInTok}[1]{{#1}}
    \newcommand{\ExtensionTok}[1]{{#1}}
    \newcommand{\PreprocessorTok}[1]{\textcolor[rgb]{0.74,0.48,0.00}{{#1}}}
    \newcommand{\AttributeTok}[1]{\textcolor[rgb]{0.49,0.56,0.16}{{#1}}}
    \newcommand{\InformationTok}[1]{\textcolor[rgb]{0.38,0.63,0.69}{\textbf{\textit{{#1}}}}}
    \newcommand{\WarningTok}[1]{\textcolor[rgb]{0.38,0.63,0.69}{\textbf{\textit{{#1}}}}}
    
    
    % Define a nice break command that doesn't care if a line doesn't already
    % exist.
    \def\br{\hspace*{\fill} \\* }
    % Math Jax compatability definitions
    \def\gt{>}
    \def\lt{<}
    % Document parameters
    \title{6.6.2 The Lasso}
    
    
    

    % Pygments definitions
    
\makeatletter
\def\PY@reset{\let\PY@it=\relax \let\PY@bf=\relax%
    \let\PY@ul=\relax \let\PY@tc=\relax%
    \let\PY@bc=\relax \let\PY@ff=\relax}
\def\PY@tok#1{\csname PY@tok@#1\endcsname}
\def\PY@toks#1+{\ifx\relax#1\empty\else%
    \PY@tok{#1}\expandafter\PY@toks\fi}
\def\PY@do#1{\PY@bc{\PY@tc{\PY@ul{%
    \PY@it{\PY@bf{\PY@ff{#1}}}}}}}
\def\PY#1#2{\PY@reset\PY@toks#1+\relax+\PY@do{#2}}

\expandafter\def\csname PY@tok@w\endcsname{\def\PY@tc##1{\textcolor[rgb]{0.73,0.73,0.73}{##1}}}
\expandafter\def\csname PY@tok@c\endcsname{\let\PY@it=\textit\def\PY@tc##1{\textcolor[rgb]{0.25,0.50,0.50}{##1}}}
\expandafter\def\csname PY@tok@cp\endcsname{\def\PY@tc##1{\textcolor[rgb]{0.74,0.48,0.00}{##1}}}
\expandafter\def\csname PY@tok@k\endcsname{\let\PY@bf=\textbf\def\PY@tc##1{\textcolor[rgb]{0.00,0.50,0.00}{##1}}}
\expandafter\def\csname PY@tok@kp\endcsname{\def\PY@tc##1{\textcolor[rgb]{0.00,0.50,0.00}{##1}}}
\expandafter\def\csname PY@tok@kt\endcsname{\def\PY@tc##1{\textcolor[rgb]{0.69,0.00,0.25}{##1}}}
\expandafter\def\csname PY@tok@o\endcsname{\def\PY@tc##1{\textcolor[rgb]{0.40,0.40,0.40}{##1}}}
\expandafter\def\csname PY@tok@ow\endcsname{\let\PY@bf=\textbf\def\PY@tc##1{\textcolor[rgb]{0.67,0.13,1.00}{##1}}}
\expandafter\def\csname PY@tok@nb\endcsname{\def\PY@tc##1{\textcolor[rgb]{0.00,0.50,0.00}{##1}}}
\expandafter\def\csname PY@tok@nf\endcsname{\def\PY@tc##1{\textcolor[rgb]{0.00,0.00,1.00}{##1}}}
\expandafter\def\csname PY@tok@nc\endcsname{\let\PY@bf=\textbf\def\PY@tc##1{\textcolor[rgb]{0.00,0.00,1.00}{##1}}}
\expandafter\def\csname PY@tok@nn\endcsname{\let\PY@bf=\textbf\def\PY@tc##1{\textcolor[rgb]{0.00,0.00,1.00}{##1}}}
\expandafter\def\csname PY@tok@ne\endcsname{\let\PY@bf=\textbf\def\PY@tc##1{\textcolor[rgb]{0.82,0.25,0.23}{##1}}}
\expandafter\def\csname PY@tok@nv\endcsname{\def\PY@tc##1{\textcolor[rgb]{0.10,0.09,0.49}{##1}}}
\expandafter\def\csname PY@tok@no\endcsname{\def\PY@tc##1{\textcolor[rgb]{0.53,0.00,0.00}{##1}}}
\expandafter\def\csname PY@tok@nl\endcsname{\def\PY@tc##1{\textcolor[rgb]{0.63,0.63,0.00}{##1}}}
\expandafter\def\csname PY@tok@ni\endcsname{\let\PY@bf=\textbf\def\PY@tc##1{\textcolor[rgb]{0.60,0.60,0.60}{##1}}}
\expandafter\def\csname PY@tok@na\endcsname{\def\PY@tc##1{\textcolor[rgb]{0.49,0.56,0.16}{##1}}}
\expandafter\def\csname PY@tok@nt\endcsname{\let\PY@bf=\textbf\def\PY@tc##1{\textcolor[rgb]{0.00,0.50,0.00}{##1}}}
\expandafter\def\csname PY@tok@nd\endcsname{\def\PY@tc##1{\textcolor[rgb]{0.67,0.13,1.00}{##1}}}
\expandafter\def\csname PY@tok@s\endcsname{\def\PY@tc##1{\textcolor[rgb]{0.73,0.13,0.13}{##1}}}
\expandafter\def\csname PY@tok@sd\endcsname{\let\PY@it=\textit\def\PY@tc##1{\textcolor[rgb]{0.73,0.13,0.13}{##1}}}
\expandafter\def\csname PY@tok@si\endcsname{\let\PY@bf=\textbf\def\PY@tc##1{\textcolor[rgb]{0.73,0.40,0.53}{##1}}}
\expandafter\def\csname PY@tok@se\endcsname{\let\PY@bf=\textbf\def\PY@tc##1{\textcolor[rgb]{0.73,0.40,0.13}{##1}}}
\expandafter\def\csname PY@tok@sr\endcsname{\def\PY@tc##1{\textcolor[rgb]{0.73,0.40,0.53}{##1}}}
\expandafter\def\csname PY@tok@ss\endcsname{\def\PY@tc##1{\textcolor[rgb]{0.10,0.09,0.49}{##1}}}
\expandafter\def\csname PY@tok@sx\endcsname{\def\PY@tc##1{\textcolor[rgb]{0.00,0.50,0.00}{##1}}}
\expandafter\def\csname PY@tok@m\endcsname{\def\PY@tc##1{\textcolor[rgb]{0.40,0.40,0.40}{##1}}}
\expandafter\def\csname PY@tok@gh\endcsname{\let\PY@bf=\textbf\def\PY@tc##1{\textcolor[rgb]{0.00,0.00,0.50}{##1}}}
\expandafter\def\csname PY@tok@gu\endcsname{\let\PY@bf=\textbf\def\PY@tc##1{\textcolor[rgb]{0.50,0.00,0.50}{##1}}}
\expandafter\def\csname PY@tok@gd\endcsname{\def\PY@tc##1{\textcolor[rgb]{0.63,0.00,0.00}{##1}}}
\expandafter\def\csname PY@tok@gi\endcsname{\def\PY@tc##1{\textcolor[rgb]{0.00,0.63,0.00}{##1}}}
\expandafter\def\csname PY@tok@gr\endcsname{\def\PY@tc##1{\textcolor[rgb]{1.00,0.00,0.00}{##1}}}
\expandafter\def\csname PY@tok@ge\endcsname{\let\PY@it=\textit}
\expandafter\def\csname PY@tok@gs\endcsname{\let\PY@bf=\textbf}
\expandafter\def\csname PY@tok@gp\endcsname{\let\PY@bf=\textbf\def\PY@tc##1{\textcolor[rgb]{0.00,0.00,0.50}{##1}}}
\expandafter\def\csname PY@tok@go\endcsname{\def\PY@tc##1{\textcolor[rgb]{0.53,0.53,0.53}{##1}}}
\expandafter\def\csname PY@tok@gt\endcsname{\def\PY@tc##1{\textcolor[rgb]{0.00,0.27,0.87}{##1}}}
\expandafter\def\csname PY@tok@err\endcsname{\def\PY@bc##1{\setlength{\fboxsep}{0pt}\fcolorbox[rgb]{1.00,0.00,0.00}{1,1,1}{\strut ##1}}}
\expandafter\def\csname PY@tok@kc\endcsname{\let\PY@bf=\textbf\def\PY@tc##1{\textcolor[rgb]{0.00,0.50,0.00}{##1}}}
\expandafter\def\csname PY@tok@kd\endcsname{\let\PY@bf=\textbf\def\PY@tc##1{\textcolor[rgb]{0.00,0.50,0.00}{##1}}}
\expandafter\def\csname PY@tok@kn\endcsname{\let\PY@bf=\textbf\def\PY@tc##1{\textcolor[rgb]{0.00,0.50,0.00}{##1}}}
\expandafter\def\csname PY@tok@kr\endcsname{\let\PY@bf=\textbf\def\PY@tc##1{\textcolor[rgb]{0.00,0.50,0.00}{##1}}}
\expandafter\def\csname PY@tok@bp\endcsname{\def\PY@tc##1{\textcolor[rgb]{0.00,0.50,0.00}{##1}}}
\expandafter\def\csname PY@tok@fm\endcsname{\def\PY@tc##1{\textcolor[rgb]{0.00,0.00,1.00}{##1}}}
\expandafter\def\csname PY@tok@vc\endcsname{\def\PY@tc##1{\textcolor[rgb]{0.10,0.09,0.49}{##1}}}
\expandafter\def\csname PY@tok@vg\endcsname{\def\PY@tc##1{\textcolor[rgb]{0.10,0.09,0.49}{##1}}}
\expandafter\def\csname PY@tok@vi\endcsname{\def\PY@tc##1{\textcolor[rgb]{0.10,0.09,0.49}{##1}}}
\expandafter\def\csname PY@tok@vm\endcsname{\def\PY@tc##1{\textcolor[rgb]{0.10,0.09,0.49}{##1}}}
\expandafter\def\csname PY@tok@sa\endcsname{\def\PY@tc##1{\textcolor[rgb]{0.73,0.13,0.13}{##1}}}
\expandafter\def\csname PY@tok@sb\endcsname{\def\PY@tc##1{\textcolor[rgb]{0.73,0.13,0.13}{##1}}}
\expandafter\def\csname PY@tok@sc\endcsname{\def\PY@tc##1{\textcolor[rgb]{0.73,0.13,0.13}{##1}}}
\expandafter\def\csname PY@tok@dl\endcsname{\def\PY@tc##1{\textcolor[rgb]{0.73,0.13,0.13}{##1}}}
\expandafter\def\csname PY@tok@s2\endcsname{\def\PY@tc##1{\textcolor[rgb]{0.73,0.13,0.13}{##1}}}
\expandafter\def\csname PY@tok@sh\endcsname{\def\PY@tc##1{\textcolor[rgb]{0.73,0.13,0.13}{##1}}}
\expandafter\def\csname PY@tok@s1\endcsname{\def\PY@tc##1{\textcolor[rgb]{0.73,0.13,0.13}{##1}}}
\expandafter\def\csname PY@tok@mb\endcsname{\def\PY@tc##1{\textcolor[rgb]{0.40,0.40,0.40}{##1}}}
\expandafter\def\csname PY@tok@mf\endcsname{\def\PY@tc##1{\textcolor[rgb]{0.40,0.40,0.40}{##1}}}
\expandafter\def\csname PY@tok@mh\endcsname{\def\PY@tc##1{\textcolor[rgb]{0.40,0.40,0.40}{##1}}}
\expandafter\def\csname PY@tok@mi\endcsname{\def\PY@tc##1{\textcolor[rgb]{0.40,0.40,0.40}{##1}}}
\expandafter\def\csname PY@tok@il\endcsname{\def\PY@tc##1{\textcolor[rgb]{0.40,0.40,0.40}{##1}}}
\expandafter\def\csname PY@tok@mo\endcsname{\def\PY@tc##1{\textcolor[rgb]{0.40,0.40,0.40}{##1}}}
\expandafter\def\csname PY@tok@ch\endcsname{\let\PY@it=\textit\def\PY@tc##1{\textcolor[rgb]{0.25,0.50,0.50}{##1}}}
\expandafter\def\csname PY@tok@cm\endcsname{\let\PY@it=\textit\def\PY@tc##1{\textcolor[rgb]{0.25,0.50,0.50}{##1}}}
\expandafter\def\csname PY@tok@cpf\endcsname{\let\PY@it=\textit\def\PY@tc##1{\textcolor[rgb]{0.25,0.50,0.50}{##1}}}
\expandafter\def\csname PY@tok@c1\endcsname{\let\PY@it=\textit\def\PY@tc##1{\textcolor[rgb]{0.25,0.50,0.50}{##1}}}
\expandafter\def\csname PY@tok@cs\endcsname{\let\PY@it=\textit\def\PY@tc##1{\textcolor[rgb]{0.25,0.50,0.50}{##1}}}

\def\PYZbs{\char`\\}
\def\PYZus{\char`\_}
\def\PYZob{\char`\{}
\def\PYZcb{\char`\}}
\def\PYZca{\char`\^}
\def\PYZam{\char`\&}
\def\PYZlt{\char`\<}
\def\PYZgt{\char`\>}
\def\PYZsh{\char`\#}
\def\PYZpc{\char`\%}
\def\PYZdl{\char`\$}
\def\PYZhy{\char`\-}
\def\PYZsq{\char`\'}
\def\PYZdq{\char`\"}
\def\PYZti{\char`\~}
% for compatibility with earlier versions
\def\PYZat{@}
\def\PYZlb{[}
\def\PYZrb{]}
\makeatother


    % Exact colors from NB
    \definecolor{incolor}{rgb}{0.0, 0.0, 0.5}
    \definecolor{outcolor}{rgb}{0.545, 0.0, 0.0}



    
    % Prevent overflowing lines due to hard-to-break entities
    \sloppy 
    % Setup hyperref package
    \hypersetup{
      breaklinks=true,  % so long urls are correctly broken across lines
      colorlinks=true,
      urlcolor=urlcolor,
      linkcolor=linkcolor,
      citecolor=citecolor,
      }
    % Slightly bigger margins than the latex defaults
    
    \geometry{verbose,tmargin=1in,bmargin=1in,lmargin=1in,rmargin=1in}
    
    

    \begin{document}
    
    
    \maketitle
    
    

    
    \begin{Verbatim}[commandchars=\\\{\}]
{\color{incolor}In [{\color{incolor}1}]:} \PY{c+c1}{\PYZsh{} conventional way to import pandas}
        \PY{k+kn}{import} \PY{n+nn}{pandas} \PY{k}{as} \PY{n+nn}{pd}
        \PY{c+c1}{\PYZsh{} conventional way to import seaborn}
        \PY{k+kn}{import} \PY{n+nn}{seaborn} \PY{k}{as} \PY{n+nn}{sns}
        \PY{c+c1}{\PYZsh{} conventional way to import numpy}
        \PY{k+kn}{import} \PY{n+nn}{numpy} \PY{k}{as} \PY{n+nn}{np}
        
        \PY{k+kn}{from} \PY{n+nn}{sklearn} \PY{k}{import} \PY{n}{metrics}
        \PY{k+kn}{import} \PY{n+nn}{matplotlib}\PY{n+nn}{.}\PY{n+nn}{pyplot} \PY{k}{as} \PY{n+nn}{plt}
        
        \PY{n}{data} \PY{o}{=} \PY{n}{pd}\PY{o}{.}\PY{n}{read\PYZus{}csv}\PY{p}{(}\PY{l+s+s2}{\PYZdq{}}\PY{l+s+s2}{https://vincentarelbundock.github.io/Rdatasets/csv/ISLR/Hitters.csv}\PY{l+s+s2}{\PYZdq{}}\PY{p}{,} \PY{n}{index\PYZus{}col}\PY{o}{=}\PY{l+m+mi}{0}\PY{p}{)}
        
        \PY{n}{data}\PY{o}{.}\PY{n}{head}\PY{p}{(}\PY{p}{)}
\end{Verbatim}


\begin{Verbatim}[commandchars=\\\{\}]
{\color{outcolor}Out[{\color{outcolor}1}]:}                    AtBat  Hits  HmRun  Runs  RBI  Walks  Years  CAtBat  CHits  \textbackslash{}
        -Andy Allanson       293    66      1    30   29     14      1     293     66   
        -Alan Ashby          315    81      7    24   38     39     14    3449    835   
        -Alvin Davis         479   130     18    66   72     76      3    1624    457   
        -Andre Dawson        496   141     20    65   78     37     11    5628   1575   
        -Andres Galarraga    321    87     10    39   42     30      2     396    101   
        
                           CHmRun  CRuns  CRBI  CWalks League Division  PutOuts  \textbackslash{}
        -Andy Allanson          1     30    29      14      A        E      446   
        -Alan Ashby            69    321   414     375      N        W      632   
        -Alvin Davis           63    224   266     263      A        W      880   
        -Andre Dawson         225    828   838     354      N        E      200   
        -Andres Galarraga      12     48    46      33      N        E      805   
        
                           Assists  Errors  Salary NewLeague  
        -Andy Allanson          33      20     NaN         A  
        -Alan Ashby             43      10   475.0         N  
        -Alvin Davis            82      14   480.0         A  
        -Andre Dawson           11       3   500.0         N  
        -Andres Galarraga       40       4    91.5         N  
\end{Verbatim}
            
    After listing the data we can see that some have missing data for their
Salary. Next drop all the rows that contain NaN data.

    \begin{Verbatim}[commandchars=\\\{\}]
{\color{incolor}In [{\color{incolor}2}]:} \PY{n}{data} \PY{o}{=} \PY{n}{data}\PY{o}{.}\PY{n}{dropna}\PY{p}{(}\PY{p}{)}
        \PY{n}{data}\PY{o}{.}\PY{n}{index}\PY{o}{.}\PY{n}{name} \PY{o}{=} \PY{l+s+s1}{\PYZsq{}}\PY{l+s+s1}{Player}\PY{l+s+s1}{\PYZsq{}}
        \PY{n}{data}\PY{o}{.}\PY{n}{head}\PY{p}{(}\PY{p}{)}
\end{Verbatim}


\begin{Verbatim}[commandchars=\\\{\}]
{\color{outcolor}Out[{\color{outcolor}2}]:}                    AtBat  Hits  HmRun  Runs  RBI  Walks  Years  CAtBat  CHits  \textbackslash{}
        Player                                                                          
        -Alan Ashby          315    81      7    24   38     39     14    3449    835   
        -Alvin Davis         479   130     18    66   72     76      3    1624    457   
        -Andre Dawson        496   141     20    65   78     37     11    5628   1575   
        -Andres Galarraga    321    87     10    39   42     30      2     396    101   
        -Alfredo Griffin     594   169      4    74   51     35     11    4408   1133   
        
                           CHmRun  CRuns  CRBI  CWalks League Division  PutOuts  \textbackslash{}
        Player                                                                    
        -Alan Ashby            69    321   414     375      N        W      632   
        -Alvin Davis           63    224   266     263      A        W      880   
        -Andre Dawson         225    828   838     354      N        E      200   
        -Andres Galarraga      12     48    46      33      N        E      805   
        -Alfredo Griffin       19    501   336     194      A        W      282   
        
                           Assists  Errors  Salary NewLeague  
        Player                                                
        -Alan Ashby             43      10   475.0         N  
        -Alvin Davis            82      14   480.0         A  
        -Andre Dawson           11       3   500.0         N  
        -Andres Galarraga       40       4    91.5         N  
        -Alfredo Griffin       421      25   750.0         A  
\end{Verbatim}
            
    In the lab they use a R funktion that we don't have in python. That
funktion automatically transforms any qualitative variables into dummy
variables. But in python we will have to do this by hand using the
following code and display infomation about the variables we converted
from strings to numbers.

    \begin{Verbatim}[commandchars=\\\{\}]
{\color{incolor}In [{\color{incolor}3}]:} \PY{n}{dummieVariables} \PY{o}{=} \PY{n}{pd}\PY{o}{.}\PY{n}{get\PYZus{}dummies}\PY{p}{(}\PY{n}{data}\PY{p}{[}\PY{p}{[}\PY{l+s+s1}{\PYZsq{}}\PY{l+s+s1}{League}\PY{l+s+s1}{\PYZsq{}}\PY{p}{,} \PY{l+s+s1}{\PYZsq{}}\PY{l+s+s1}{Division}\PY{l+s+s1}{\PYZsq{}}\PY{p}{,} \PY{l+s+s1}{\PYZsq{}}\PY{l+s+s1}{NewLeague}\PY{l+s+s1}{\PYZsq{}}\PY{p}{]}\PY{p}{]}\PY{p}{)}
        \PY{n}{dummieVariables}\PY{o}{.}\PY{n}{info}\PY{p}{(}\PY{p}{)}
        \PY{n+nb}{print}\PY{p}{(}\PY{n}{dummieVariables}\PY{o}{.}\PY{n}{head}\PY{p}{(}\PY{p}{)}\PY{p}{)}
\end{Verbatim}


    \begin{Verbatim}[commandchars=\\\{\}]
<class 'pandas.core.frame.DataFrame'>
Index: 263 entries, -Alan Ashby to -Willie Wilson
Data columns (total 6 columns):
League\_A       263 non-null uint8
League\_N       263 non-null uint8
Division\_E     263 non-null uint8
Division\_W     263 non-null uint8
NewLeague\_A    263 non-null uint8
NewLeague\_N    263 non-null uint8
dtypes: uint8(6)
memory usage: 3.6+ KB
                   League\_A  League\_N  Division\_E  Division\_W  NewLeague\_A  \textbackslash{}
Player                                                                       
-Alan Ashby               0         1           0           1            0   
-Alvin Davis              1         0           0           1            1   
-Andre Dawson             0         1           1           0            0   
-Andres Galarraga         0         1           1           0            0   
-Alfredo Griffin          1         0           0           1            1   

                   NewLeague\_N  
Player                          
-Alan Ashby                  1  
-Alvin Davis                 0  
-Andre Dawson                1  
-Andres Galarraga            1  
-Alfredo Griffin             0  

    \end{Verbatim}

    Next we must remove the columns with our independent variable (Salary),
and columns for which we created dummy variables and reinterduce them
into our predictors. This is do so the data fit the data in the book.

    \begin{Verbatim}[commandchars=\\\{\}]
{\color{incolor}In [{\color{incolor}4}]:} \PY{n}{y} \PY{o}{=} \PY{n}{data}\PY{o}{.}\PY{n}{Salary}
        \PY{n}{X\PYZus{}} \PY{o}{=} \PY{n}{data}\PY{o}{.}\PY{n}{drop}\PY{p}{(}\PY{p}{[}\PY{l+s+s1}{\PYZsq{}}\PY{l+s+s1}{Salary}\PY{l+s+s1}{\PYZsq{}}\PY{p}{,} \PY{l+s+s1}{\PYZsq{}}\PY{l+s+s1}{League}\PY{l+s+s1}{\PYZsq{}}\PY{p}{,} \PY{l+s+s1}{\PYZsq{}}\PY{l+s+s1}{Division}\PY{l+s+s1}{\PYZsq{}}\PY{p}{,} \PY{l+s+s1}{\PYZsq{}}\PY{l+s+s1}{NewLeague}\PY{l+s+s1}{\PYZsq{}}\PY{p}{]}\PY{p}{,} \PY{n}{axis}\PY{o}{=}\PY{l+m+mi}{1}\PY{p}{)}\PY{o}{.}\PY{n}{astype}\PY{p}{(}\PY{l+s+s1}{\PYZsq{}}\PY{l+s+s1}{float64}\PY{l+s+s1}{\PYZsq{}}\PY{p}{)}
        
        \PY{n}{X} \PY{o}{=} \PY{n}{pd}\PY{o}{.}\PY{n}{concat}\PY{p}{(}\PY{p}{[}\PY{n}{X\PYZus{}}\PY{p}{,} \PY{n}{dummieVariables}\PY{p}{[}\PY{p}{[}\PY{l+s+s1}{\PYZsq{}}\PY{l+s+s1}{League\PYZus{}N}\PY{l+s+s1}{\PYZsq{}}\PY{p}{,} \PY{l+s+s1}{\PYZsq{}}\PY{l+s+s1}{Division\PYZus{}W}\PY{l+s+s1}{\PYZsq{}}\PY{p}{,} \PY{l+s+s1}{\PYZsq{}}\PY{l+s+s1}{NewLeague\PYZus{}N}\PY{l+s+s1}{\PYZsq{}}\PY{p}{]}\PY{p}{]}\PY{p}{]}\PY{p}{,} \PY{n}{axis}\PY{o}{=}\PY{l+m+mi}{1}\PY{p}{)}
        \PY{n}{X}\PY{o}{.}\PY{n}{head}\PY{p}{(}\PY{p}{)}
\end{Verbatim}


\begin{Verbatim}[commandchars=\\\{\}]
{\color{outcolor}Out[{\color{outcolor}4}]:}                    AtBat   Hits  HmRun  Runs   RBI  Walks  Years  CAtBat  \textbackslash{}
        Player                                                                     
        -Alan Ashby        315.0   81.0    7.0  24.0  38.0   39.0   14.0  3449.0   
        -Alvin Davis       479.0  130.0   18.0  66.0  72.0   76.0    3.0  1624.0   
        -Andre Dawson      496.0  141.0   20.0  65.0  78.0   37.0   11.0  5628.0   
        -Andres Galarraga  321.0   87.0   10.0  39.0  42.0   30.0    2.0   396.0   
        -Alfredo Griffin   594.0  169.0    4.0  74.0  51.0   35.0   11.0  4408.0   
        
                            CHits  CHmRun  CRuns   CRBI  CWalks  PutOuts  Assists  \textbackslash{}
        Player                                                                      
        -Alan Ashby         835.0    69.0  321.0  414.0   375.0    632.0     43.0   
        -Alvin Davis        457.0    63.0  224.0  266.0   263.0    880.0     82.0   
        -Andre Dawson      1575.0   225.0  828.0  838.0   354.0    200.0     11.0   
        -Andres Galarraga   101.0    12.0   48.0   46.0    33.0    805.0     40.0   
        -Alfredo Griffin   1133.0    19.0  501.0  336.0   194.0    282.0    421.0   
        
                           Errors  League\_N  Division\_W  NewLeague\_N  
        Player                                                        
        -Alan Ashby          10.0         1           1            1  
        -Alvin Davis         14.0         0           1            0  
        -Andre Dawson         3.0         1           0            1  
        -Andres Galarraga     4.0         1           0            1  
        -Alfredo Griffin     25.0         0           1            0  
\end{Verbatim}
            
    Now we spilt the data into a tranining and testing set like the book.

    \begin{Verbatim}[commandchars=\\\{\}]
{\color{incolor}In [{\color{incolor}5}]:} \PY{k+kn}{from} \PY{n+nn}{sklearn}\PY{n+nn}{.}\PY{n+nn}{model\PYZus{}selection} \PY{k}{import} \PY{n}{train\PYZus{}test\PYZus{}split}
        \PY{n}{X\PYZus{}train}\PY{p}{,} \PY{n}{X\PYZus{}test}\PY{p}{,} \PY{n}{y\PYZus{}train}\PY{p}{,} \PY{n}{y\PYZus{}test} \PY{o}{=} \PY{n}{train\PYZus{}test\PYZus{}split}\PY{p}{(}\PY{n}{X}\PY{p}{,} \PY{n}{y}\PY{p}{,} \PY{n}{random\PYZus{}state}\PY{o}{=}\PY{l+m+mi}{10}\PY{p}{)}
\end{Verbatim}


    \begin{Verbatim}[commandchars=\\\{\}]
{\color{incolor}In [{\color{incolor}6}]:} \PY{c+c1}{\PYZsh{}\PYZsh{}Ploting code from scikit\PYZhy{}learn.org just changed to use lasso.}
        \PY{c+c1}{\PYZsh{}\PYZsh{} http://scikit\PYZhy{}learn.org/stable/auto\PYZus{}examples/linear\PYZus{}model/plot\PYZus{}ridge\PYZus{}path.html\PYZsh{}sphx\PYZhy{}glr\PYZhy{}auto\PYZhy{}examples\PYZhy{}linear\PYZhy{}model\PYZhy{}plot\PYZhy{}ridge\PYZhy{}path\PYZhy{}py}
        \PY{k+kn}{from} \PY{n+nn}{sklearn}\PY{n+nn}{.}\PY{n+nn}{preprocessing} \PY{k}{import} \PY{n}{scale}
        \PY{k+kn}{from} \PY{n+nn}{sklearn}\PY{n+nn}{.}\PY{n+nn}{linear\PYZus{}model} \PY{k}{import} \PY{n}{Lasso}
        
        \PY{n}{alphas} \PY{o}{=} \PY{l+m+mi}{10}\PY{o}{*}\PY{o}{*}\PY{n}{np}\PY{o}{.}\PY{n}{linspace}\PY{p}{(}\PY{l+m+mi}{10}\PY{p}{,}\PY{o}{\PYZhy{}}\PY{l+m+mi}{2}\PY{p}{,}\PY{l+m+mi}{100}\PY{p}{)}\PY{o}{*}\PY{l+m+mf}{0.5}
        \PY{n}{lasso} \PY{o}{=} \PY{n}{Lasso}\PY{p}{(}\PY{n}{max\PYZus{}iter}\PY{o}{=}\PY{l+m+mi}{100000}\PY{p}{)}
        \PY{n}{coefs} \PY{o}{=} \PY{p}{[}\PY{p}{]}
        
        \PY{k}{for} \PY{n}{a} \PY{o+ow}{in} \PY{n}{alphas}\PY{o}{*}\PY{l+m+mi}{2}\PY{p}{:}
            \PY{n}{lasso}\PY{o}{.}\PY{n}{set\PYZus{}params}\PY{p}{(}\PY{n}{alpha}\PY{o}{=}\PY{n}{a}\PY{p}{)}
            \PY{n}{lasso}\PY{o}{.}\PY{n}{fit}\PY{p}{(}\PY{n}{X\PYZus{}train}\PY{p}{,} \PY{n}{y\PYZus{}train}\PY{p}{)}
            \PY{n}{coefs}\PY{o}{.}\PY{n}{append}\PY{p}{(}\PY{n}{lasso}\PY{o}{.}\PY{n}{coef\PYZus{}}\PY{p}{)}
        
        \PY{n}{ax} \PY{o}{=} \PY{n}{plt}\PY{o}{.}\PY{n}{gca}\PY{p}{(}\PY{p}{)}
        \PY{n}{ax}\PY{o}{.}\PY{n}{plot}\PY{p}{(}\PY{n}{alphas}\PY{o}{*}\PY{l+m+mi}{2}\PY{p}{,} \PY{n}{coefs}\PY{p}{)}
        \PY{n}{ax}\PY{o}{.}\PY{n}{set\PYZus{}xscale}\PY{p}{(}\PY{l+s+s1}{\PYZsq{}}\PY{l+s+s1}{log}\PY{l+s+s1}{\PYZsq{}}\PY{p}{)}
        \PY{n}{ax}\PY{o}{.}\PY{n}{set\PYZus{}xlim}\PY{p}{(}\PY{n}{ax}\PY{o}{.}\PY{n}{get\PYZus{}xlim}\PY{p}{(}\PY{p}{)}\PY{p}{[}\PY{p}{:}\PY{p}{:}\PY{o}{\PYZhy{}}\PY{l+m+mi}{1}\PY{p}{]}\PY{p}{)}  \PY{c+c1}{\PYZsh{} reverse axis}
        \PY{n}{plt}\PY{o}{.}\PY{n}{axis}\PY{p}{(}\PY{l+s+s1}{\PYZsq{}}\PY{l+s+s1}{tight}\PY{l+s+s1}{\PYZsq{}}\PY{p}{)}
        \PY{n}{plt}\PY{o}{.}\PY{n}{xlabel}\PY{p}{(}\PY{l+s+s1}{\PYZsq{}}\PY{l+s+s1}{alpha}\PY{l+s+s1}{\PYZsq{}}\PY{p}{)}
        \PY{n}{plt}\PY{o}{.}\PY{n}{ylabel}\PY{p}{(}\PY{l+s+s1}{\PYZsq{}}\PY{l+s+s1}{weights}\PY{l+s+s1}{\PYZsq{}}\PY{p}{)}
        \PY{n}{plt}\PY{o}{.}\PY{n}{title}\PY{p}{(}\PY{l+s+s1}{\PYZsq{}}\PY{l+s+s1}{Lasso coefficients as a function of the regularization}\PY{l+s+s1}{\PYZsq{}}\PY{p}{)}\PY{p}{;}
\end{Verbatim}


    \begin{center}
    \adjustimage{max size={0.9\linewidth}{0.9\paperheight}}{output_9_0.png}
    \end{center}
    { \hspace*{\fill} \\}
    
    Next we try to get the best lamda using cross validation.

    \begin{Verbatim}[commandchars=\\\{\}]
{\color{incolor}In [{\color{incolor}7}]:} \PY{k+kn}{from} \PY{n+nn}{sklearn}\PY{n+nn}{.}\PY{n+nn}{linear\PYZus{}model} \PY{k}{import} \PY{n}{LassoCV}
        \PY{n}{alphas} \PY{o}{=} \PY{l+m+mi}{10}\PY{o}{*}\PY{o}{*}\PY{n}{np}\PY{o}{.}\PY{n}{linspace}\PY{p}{(}\PY{l+m+mi}{100}\PY{p}{,}\PY{o}{\PYZhy{}}\PY{l+m+mi}{20}\PY{p}{,}\PY{l+m+mi}{100}\PY{p}{)}
        \PY{n}{lassoCV} \PY{o}{=} \PY{n}{LassoCV}\PY{p}{(}\PY{n}{alphas}\PY{o}{=}\PY{n}{alphas}\PY{p}{)}
        \PY{n}{lassoCV}\PY{o}{.}\PY{n}{fit}\PY{p}{(}\PY{n}{X\PYZus{}train}\PY{p}{,} \PY{n}{y\PYZus{}train}\PY{p}{)}
        \PY{n}{lassoCV}\PY{o}{.}\PY{n}{alpha\PYZus{}}
\end{Verbatim}


    \begin{Verbatim}[commandchars=\\\{\}]
C:\textbackslash{}Users\textbackslash{}entvex\textbackslash{}Anaconda3\textbackslash{}lib\textbackslash{}site-packages\textbackslash{}sklearn\textbackslash{}linear\_model\textbackslash{}coordinate\_descent.py:491: ConvergenceWarning: Objective did not converge. You might want to increase the number of iterations. Fitting data with very small alpha may cause precision problems.
  ConvergenceWarning)

    \end{Verbatim}

\begin{Verbatim}[commandchars=\\\{\}]
{\color{outcolor}Out[{\color{outcolor}7}]:} 1072.2672220103254
\end{Verbatim}
            
    \begin{Verbatim}[commandchars=\\\{\}]
{\color{incolor}In [{\color{incolor}8}]:} \PY{n}{pd}\PY{o}{.}\PY{n}{Series}\PY{p}{(}\PY{n}{lassoCV}\PY{o}{.}\PY{n}{coef\PYZus{}}\PY{o}{.}\PY{n}{flatten}\PY{p}{(}\PY{p}{)}\PY{p}{,} \PY{n}{index}\PY{o}{=}\PY{n}{X}\PY{o}{.}\PY{n}{columns}\PY{p}{)}
\end{Verbatim}


\begin{Verbatim}[commandchars=\\\{\}]
{\color{outcolor}Out[{\color{outcolor}8}]:} AtBat          0.729684
        Hits           0.000000
        HmRun         -0.000000
        Runs           0.000000
        RBI            0.000000
        Walks          0.000000
        Years         -0.000000
        CAtBat        -0.241005
        CHits          0.705141
        CHmRun         0.000000
        CRuns          0.579498
        CRBI           0.442905
        CWalks        -0.000000
        PutOuts        0.232895
        Assists        0.000000
        Errors        -0.000000
        League\_N       0.000000
        Division\_W    -0.000000
        NewLeague\_N    0.000000
        dtype: float64
\end{Verbatim}
            

    % Add a bibliography block to the postdoc
    
    
    
    \end{document}
