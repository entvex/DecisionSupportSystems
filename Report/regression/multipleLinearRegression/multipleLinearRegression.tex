\section{Multiple Linear Regression}\label{sc:multipleLinearRegression}

\subsection{Theory}

Basic theory for simple and multiple lin regs here. From the slides or book.

Simple Linear Regression is used to make linear models of data. It has a response Y on the basis of a single predictor variable X. We can write it as
$Y = \beta_0 + \beta_1 X_1 + \epsilon_i$.
$ \beta_0 + \beta_1 $ are unknown and to get a response, we must use data to estimate the coefficients.
$(x_1, y_1), (x_2, y_2), . . . , (x_n, y_n)$
represent n observation pairs, each of which consists of a measurement of X and a measurement of Y. The drawback of this method is that only a single predictor variable is used and often have more.
 In cases where we want examined the relationship between multiple predictor variables we use Multiple Linear Regression. The model takes the following form $Y = \beta_0 + \beta_1 X_1 + ... + \beta_n X_n + \epsilon_i$

To obtain the Coefficients in the model we use the least squares method to minimize the sum of squared residuals. We pick $\beta_0, \beta_1, ... \beta_p$ to to minimize the sum of squared residuals.
$$RSS = \sum (y - \hat{y})^2 = \sum( y_i - \hat{\beta_0} - \hat{\beta_1}x_\textit{i}1 + \hat{\beta_2}x_\textit{i}2 - \dots - \hat{\beta_p}x_\textit{i}p )^2$$

To evaluated the model we can use RSE (residual standard error)


\subsection{Results}

LAB 3.6.2 + 3.6.3

\subsection{Conclusion}