% Dokumentklassen sættes til memoir.
% Manual: http://ctan.org/tex-archive/macros/latex/contrib/memoir/memman.pdf
%\documentclass[a4paper,11pt,twoside,openright]{memoir}
\setlrmarginsandblock{*}{2.5cm}{0.75} % højre og venstre 
\setulmarginsandblock{3cm}{*}{1.2} % top og bund 
\checkandfixthelayout[nearest] % specifikt valg af højde algoritme

%Styrer hvordan nye afsnit håndteres
%https://www.sharelatex.com/learn/Paragraph_formatting#Reference_guide
\parindent=1em %Start på nyt afsnit rykkes ind
\parskip=0.5em %Mellemrum mellem afsnit
 
% Danske udtryk (fx figur og tabel) samt dansk orddeling og fonte med
% danske tegn. Hvis LaTeX brokker sig over æ, ø og å skal du udskifte
% "utf8" med "latin1" eller "applemac". 
\usepackage[utf8]{inputenc}
\usepackage[english]{babel}
\usepackage[T1]{fontenc}
\usepackage{mflogo}
\usepackage{titlesec}
\titleformat{\chapter}[hang] 
{\normalfont\huge\bfseries}{\chaptertitlename\ \thechapter:}{1em}{} 
\titlespacing*{\chapter}{0pt}{-50pt}{40pt}

%bruges til at fastgøre billeder hvor man vil have dem ved brug af H
\usepackage{float}

%sexy pdf'er
%\usepackage[export]{adjustbox}
\usepackage{pdfpages}
\usepackage{pdflscape}

%Kompakte lister
\usepackage{paralist}
 
% Matematisk udtryk, fede symboler, theoremer og fancy ting (fx kædebrøker)
\usepackage{amsmath,amssymb}
\usepackage{bm}
\usepackage{amsthm}
\usepackage{mathtools}

% Fancy ting med enheder og datatabeller. Læs manualen til pakken
% Manual: http://www.ctan.org/tex-archive/macros/latex/contrib/siunitx/siunitx.pdf
\usepackage{siunitx}
 
%Fancy headers, 
%Manual: https://www.sharelatex.com/learn/Headers_and_footers
\let\footruleskip\undefined
\usepackage{fancyhdr}
\pagestyle{fancy}

 
% Indsættelse af grafik. og man kan rotere tekst in line
\usepackage{graphicx} 
\usepackage{fix-cm} 
\usepackage{soul}
\sodef\an{}{0.13em}{0em}{0em} \sodef\ann{}{0.13em}{0.5em}{0em}
 

%Fancy tabeller.
%\usepackage[table]{xcolor}
\usepackage{multirow}
\usepackage{rotating} %sidewaystables!
\usepackage{longtable} %tables spanning multible pages.
\usepackage{tablefootnote} %for at indstætte fornoter i tabeller.
\usepackage{hhline} %Fixer farvede felter
\usepackage{ltxtable} %Longtabular X
\usepackage{tabularx} %Med dynamisk bredte

%URL fodnoter
\usepackage{url}

% Reaktionsskemaer. Læs manualen for at se eksempler.
% Manual: http://www.ctan.org/tex-archive/macros/latex/contrib/mhchem/mhchem.pdf
\usepackage[version=3]{mhchem}

%Lav chapter clickable og fjern border
\usepackage{hyperref}
\hypersetup{
    colorlinks,
    citecolor=black,
    filecolor=black,
    linkcolor=black,
    urlcolor=black
}

%Table of contents settings
\setsecnumdepth{subsection} % organisational level that receives a numbers
\settocdepth{subsection}   % print table of  for level 3

%Til programkode
\usepackage{listings}
\usepackage{color}

\definecolor{dkgreen}{rgb}{0,0.6,0}
\definecolor{gray}{rgb}{0.5,0.5,0.5}
\definecolor{mauve}{rgb}{0.58,0,0.82}
 
 % Added to use LaTeX From jupiternotebook
 \usepackage{fancyvrb}
 \definecolor{incolor}{rgb}{0.0, 0.0, 0.5}
 \definecolor{outcolor}{rgb}{0.545, 0.0, 0.0}
 
\lstset{ 
  language=C++,                % the language of the code
  basicstyle=\footnotesize,           % the size of the fonts that are used for the code
  numbers=left,                   % where to put the line-numbers
  numberstyle=\tiny\color{gray},  % the style that is used for the line-numbers
  stepnumber=1,                   % the step between two line-numbers. If it's 1, each line 
                                  % will be numbered
  numbersep=5pt,                  % how far the line-numbers are from the code
  backgroundcolor=\color{white},      % choose the background color. You must add \usepackage{color}
  showspaces=false,               % show spaces adding particular underscores
  showstringspaces=false,         % underline spaces within strings
  showtabs=false,                 % show tabs within strings adding particular underscores
  frame=single,                   % adds a frame around the code
  rulecolor=\color{black},        % if not set, the frame-color may be changed on line-breaks within not-black text (e.g. commens (green here))
  tabsize=2,                      % sets default tabsize to 2 spaces
  captionpos=b,                   % sets the caption-position to bottom
  breaklines=true,                % sets automatic line breaking
  breakatwhitespace=false,        % sets if automatic breaks should only happen at whitespace
  title=\lstname,                   % show the filename of files included with \lstinputlisting;
                                  % also try caption instead of title
  keywordstyle=\color{blue},          % keyword style
  commentstyle=\color{dkgreen},       % comment style
  stringstyle=\color{mauve},         % string literal style
  escapeinside={\%*}{*)},            % if you want to add LaTeX within your code
  morekeywords={*,...},               % if you want to add more keywords to the set
  rangeprefix=//----------,			%Used for sexy code includes
  rangesuffix=----------,			%---||---
  includerangemarker=false,			%---||---
  literate=
  {á}{{\'a}}1 {é}{{\'e}}1 {í}{{\'i}}1 {ó}{{\'o}}1 {ú}{{\'u}}1
  {Á}{{\'A}}1 {É}{{\'E}}1 {Í}{{\'I}}1 {Ó}{{\'O}}1 {Ú}{{\'U}}1
  {à}{{\`a}}1 {è}{{\`e}}1 {ì}{{\`i}}1 {ò}{{\`o}}1 {ù}{{\`u}}1
  {À}{{\`A}}1 {È}{{\'E}}1 {Ì}{{\`I}}1 {Ò}{{\`O}}1 {Ù}{{\`U}}1
  {ä}{{\"a}}1 {ë}{{\"e}}1 {ï}{{\"i}}1 {ö}{{\"o}}1 {ü}{{\"u}}1
  {Ä}{{\"A}}1 {Ë}{{\"E}}1 {Ï}{{\"I}}1 {Ö}{{\"O}}1 {Ü}{{\"U}}1
  {â}{{\^a}}1 {ê}{{\^e}}1 {î}{{\^i}}1 {ô}{{\^o}}1 {û}{{\^u}}1
  {Â}{{\^A}}1 {Ê}{{\^E}}1 {Î}{{\^I}}1 {Ô}{{\^O}}1 {Û}{{\^U}}1
  {œ}{{\oe}}1 {Œ}{{\OE}}1 {æ}{{\ae}}1 {Æ}{{\AE}}1 {ß}{{\ss}}1
  {ç}{{\c c}}1 {Ç}{{\c C}}1 {ø}{{\o}}1 {å}{{\r a}}1 {Å}{{\r A}}1
  {€}{{\EUR}}1 {£}{{\pounds}}1
}

%Til at udregne forskel mellem sider, brug \pagedifference{A}{B} mellem to labels A og B.
\usepackage{refcount}
\newcommand{\pagedifference}[2]{%
  \number\numexpr\getpagerefnumber{#2}-\getpagerefnumber{#1}\relax}
 
%Til at lave referencer med:
\usepackage{cite}

%Til at lave eksterne \ref til \labels
\usepackage{xr}

%Til at lave \Beam (DC symbol)
\usepackage{marvosym}

%Forsøg på nice lister i tabeller
\usepackage[shortlabels]{enumitem}

\newenvironment{packed_enum}{
\begin{enumerate}[1., topsep=0pt, nosep, partopsep=0pt, itemsep=0pt, parsep=0pt]
}{\end{enumerate}}

\newenvironment{packed_item}{
\begin{itemize}[•, topsep=0pt, nosep, partopsep=0pt, itemsep=0pt, parsep=0pt]
}{\end{itemize}}

%Lækker kommando til at skrive I2C flot uden at bruge \textsuperscript hver gang:
\newcommand*{\IIC}{\texorpdfstring{I\textsuperscript{2}C }{I2C}}

%Lækker kommando til ref. -> \ref{input} \nameref{input} på side \pageref{input}
\newcommand*{\myRef}[1] {\ref{#1} \nameref{#1} på side \pageref{#1}}

%Lorem ipsum
\usepackage{lipsum}


\usepackage{longtable}
\usepackage{array} % for extrarowheight

%Juicy columntypes - http://tex.stackexchange.com/questions/12703/how-to-create-fixed-width-table-columns-with-text-raggedright-centered-raggedlef
\newcolumntype{L}[1]{>{\raggedright\let\newline\\\arraybackslash\hspace{0pt}}p{#1}}
\newcolumntype{C}[1]{>{\centering\let\newline\\\arraybackslash\hspace{0pt}}p{#1}}
\newcolumntype{R}[1]{>{\raggedleft\let\newline\\\arraybackslash\hspace{0pt}}p{#1}}
\newcolumntype{Z}{>{\raggedright\arraybackslash}X}

%Dejlig kommando til at få nye kapitler på højre side
\newcommand*\cleartorightpage{%
	\clearpage
 	\checkoddpage
	\ifoddpage
  		%do nothing
	\else
		\thispagestyle{empty}
		\mbox{}
 		\clearpage
	\fi
}


%Hacky løsning til at ordne indholdsfortegnelsen.. Why memoir class.. WHY??!
\renewcommand*{\cftdotsep}{1}
\setpnumwidth{3em}
\setrmarg{4em}

%Bugfix til Longtables
\makeatletter
\def\LT@start{%
  \let\LT@start\endgraf
  \endgraf\penalty\z@\vskip\LTpre
  \dimen@\pagetotal
  \advance\dimen@ \ht\ifvoid\LT@firsthead\LT@head\else\LT@firsthead\fi
  \advance\dimen@ \dp\ifvoid\LT@firsthead\LT@head\else\LT@firsthead\fi
  \advance\dimen@ \ht\LT@foot
  \edef\restore@vbadness{\vbadness\the\vbadness\relax}% (added)
  \vbadness=\@M % (added)
  \dimen@ii\vfuzz
  \vfuzz\maxdimen
    \setbox\tw@\copy\z@
    \setbox\tw@\vsplit\tw@ to \ht\@arstrutbox
    \setbox\tw@\vbox{\unvbox\tw@}%
  \vfuzz\dimen@ii
  \restore@vbadness % (added)
  \advance\dimen@ \ht
        \ifdim\ht\@arstrutbox>\ht\tw@\@arstrutbox\else\tw@\fi
  \advance\dimen@\dp
        \ifdim\dp\@arstrutbox>\dp\tw@\@arstrutbox\else\tw@\fi
  \advance\dimen@ -\pagegoal
  \ifdim \dimen@>\z@\vfil\break\fi
      \global\@colroom\@colht
  \ifvoid\LT@foot\else
    \advance\vsize-\ht\LT@foot
    \global\advance\@colroom-\ht\LT@foot
    \dimen@\pagegoal\advance\dimen@-\ht\LT@foot\pagegoal\dimen@
    \maxdepth\z@
  \fi
  \ifvoid\LT@firsthead\copy\LT@head\else\box\LT@firsthead\fi\nobreak
  \output{\LT@output}%
}
\makeatother