\chapter{Introduction} \label{ch:introduction}

This document will cover the concepts covered in the course Decision Support Systems and is segmented by lecture topic covering a broader catagory of supervised or unsupervised learning. This report is intended to give an overview of the methods discussed during lectures, but not to be a definitive guide to machine learning methods.

Decision support systems are a field of study that focuses on using statistical models to provide understanding of the context that is free of cognitive biases and sufficiently numerical in nature to allow computer agents to use them as the basis for making decisions despite uncertainty. As problems become more complex, the capacity of humans to gain oversight and understanding of the situation drops, and computers start needing more complex heuristics to make decisions. In both cases relying on humans adds weakness to the solution as they tend to use few simple heuristics to make decisions, which often leads to severe and often repeated errors\footnote{\cite{Tversky1975}}.

When it comes to AI and machine learning, decision support systems can be applied to allow the machine to handle uncertainty in a systematic manner. The machine will then be able to create highly complex decision models based on available data, in ways that no human programmer could manage. Many of the topics covered in the course rely heavily on the application of statistical methods for producing their results. As such many of the fallacies an author might make with statistics also applies to machine learning methods, as the foundation is the actual statistical measurement techniques. If a statistics model is poorly designed by an individual, it might as well perform badly in a computer setting.

This could also lead to illusory correlation, where a certain association between variables in a model correlate with each-other, thus a causation is assumed if one blindly trusts the statistical model. Statistics and machine learning are extremely useful tools and will continue to see use in modern society and technology, but it is important to keep in mind the mathematics behind and use common sense when evaluating the results.
